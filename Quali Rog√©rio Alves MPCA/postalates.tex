Received: from ATHENA.MIT.EDU by po6.MIT.EDU (5.61/4.7) id AA15720; Tue, 9 Jun 92 14:06:42 -0400
Received: from THYME.LCS.MIT.EDU by Athena.MIT.EDU with SMTP
	id AA05935; Tue, 9 Jun 92 14:06:34 EDT
Received: from FLY.LCS.MIT.EDU by thyme.LCS.MIT.EDU via TCP with SMTP
	id AA01869; Tue, 9 Jun 92 14:06:24 EDT
Received: by fly.lcs.mit.edu.lcs.mit.edu (4.1/SMI-4.1)
	id AA06070; Tue, 9 Jun 92 14:09:33 EDT
Date: Tue, 9 Jun 92 14:09:33 EDT
Message-Id: <9206091809.AA06070@fly.lcs.mit.edu.lcs.mit.edu>
From: Semyon Dukach <dukach@lcs.mit.edu>
Sender: dukach@fly.lcs.mit.edu
To: sao@Athena.MIT.EDU
Cc: sao@Athena.MIT.EDU
In-Reply-To: sao@athena.mit.edu's message of Tue, 09 Jun 92 10:10:05 EDT <9206091410.AA08997@cascade>
Subject: SNPP protocol

Here's a copy of my paper on SNPP.  If you want to talk about it, you are
welcome to stop by my office in 501 LCS anytime.

/semyon

% -*- Mode: LaTeX -*-
\documentstyle[12pt]{article}
\begin{document}

\begin{center}
{\large {\em \bf SNPP:} A Simple Network Payment Protocol \\}
\bigskip
\bigskip
{\em Semyon Dukach\\}
\bigskip
NE43-501\\
Laboratory for Computer Science\\
Massachusetts Institute of Technology\\
Cambridge, MA 02139\\
dukach@lcs.mit.edu\\
(617)253-6079\\
\end{center}

\bigskip
\bigskip

\begin{abstract}
A protocol is proposed to securely implement payment transactions
between mutually distrustful parties.  This protocol is designed to
operate over an open network, and can be implemented using currently
available encryption technology.
\end{abstract}

\section{Introduction}

When the Internet was first created, many exciting services and
applications were envisioned.  It was widely believed that the network
would be used to fulfill diverse needs and to provide many new
capabilities for users around the world.  Powerful services providing
news, shopping, entertainment, and remote computation and storage were
believed to be around the corner.  Yet despite the fact that today's Internet
has grown to sufficient scope and speed, very few advanced services are
available at the moment.  The large majority of traffic still consists of
simple electronic mail and ``news'' delivery, remote logins, and file
transfers.

We believe that a major reason why advanced applications
have not flourished is the lack of a practical payment facility.
Without the ability to efficiently exchange money, there is
insufficient motivation for the development of production-quality
software.  In addition, since further growth of the Internet seems
much less likely to rely on government funding, payment mechanisms are
also needed to enable link owners to efficiently charge their users.

Informal {\it ad hoc\/} financial transactions have already begun to occur on
the Internet.  Most of these, such as sending credit card numbers
through e-mail, are extremely insecure and inefficient.  It is instead
desirable to allow individuals with access to the Internet to pay for
goods and services directly.

This paper describes SNPP, a simple payment protocol for secure
transactions over an open network.  A logical verification of the
protocol is included, as well as a status report on its
implementation.

\section{Goals}

Our payment protocol has been designed to meet the following
goals:
\begin{itemize}
\item It should provide a secure means
for payment transactions between parties over an open network.

\item The parties
should not need to trust each other, only their respective banks.

\item A thief with the
ability to arbitrarily view, store, and replace messages among all parties
should at most cause denial of service, and should not be able to
embezzle funds, even if allied with one of the parties.

\item   A mechanism should be provided for
``holding'' the funds during the sensitive period of a transaction, so that
no money is handed over until a product is received, yet no product is
sent out until payment is assured.

\item The protocol should be fully
partitionable, in the sense that a customer should only deal with his merchant,
and a merchant only with her bank. Therefore at each point, further
verification could be postponed for batch processing at the risk of the
appropriate party, and without a separate procedure at the beginning of the
authorization chain.

\end{itemize}

For our initial implementation, the basic requirements are library calls
providing UDP datagram
communication, and DES encryption.  In addition, several support
functions such as reliable key distribution, stable storage for used
transaction numbers, and workstation authentication will be necessary for
an implementation secure enough for experimenting with non-trivial amounts
of real currency.

\section{Background}

In order to provide the ability to transfer payments in real time, a
mechanism is necessary for secure communication.  In 1978, Needham and
Schroeder \cite{kn:needham78} defined protocols for establishing secure interactive
connections by using an authentication server.
The Kerberos \cite{kn:steiner88} authentication system
extends the Needham and Schroeder algorithm to multiple authentication
servers.  Authorization and accounting schemes can be built on top of the
authentication interface that Kerberos provides.  The latest version
of Kerberos also supports an extension defined by Neuman
\cite{kn:neuman91} as
{\it Restricted Proxies}.  The recipient of such a proxy is allowed to act
with the same rights and privileges as the grantor of the proxy, subject to
an arbitrary set of restrictions.  Neuman briefly outlines how a payment
protocol can be implemented on top of restricted proxies.

While restricted proxies and Kerberos may prove to be a desirable
platform in future implementations, the protocol we are currently
proposing has been designed from scratch, using UDP for communication,
and DES for encryption.  This has enabled us to give closer
consideration to the fundamental security and performance tradeoffs
inherent in the problem of open network payments.  In addition, direct
implementations allow the inclusion of several desirable properties,
such as anonymity between the buyer and the seller, and
partitionability of payment verification.

A network payment protocol is only as secure as the end workstations.  In
particular, a user has to be able to authenticate a workstation before
divulging his key.
As shown in Abadi {\it et al} \cite{kn:abadi90}, workstation authentication
can only be
achieved with a smart card which minimally possesses a self-powered clock,
an internal encryption and decryption capability, and either a keyboard or
a display.  In their 1990 report, Bos and Chaum \cite{kn:bos90} describe an
electronic payment system which employs smart cards capable of DES but not
RSA encryption to authenticate host workstations.  Since our goal has been to
design a payment system for the immediate future which does not require any
special hardware,
the protocol assumes that the user's end
workstation is secure, without making any assumptions about the security of
the network.

In order to verify the correctness of our protocol, we make use of
{\it A Logic of Authentication}, developed by Burrows, Abadi, and Needham
\cite{kn:burrows89}.  The proof of the correctness of SNPP outlined below
is analogous to the verification of Kerberos given as an example in
their paper.

\section{Summary of the Protocol}

In order to act as either a customer or a merchant, a user must first
open an account with a trusted bank and deposit some funds.  At the
time an account is opened, a key is generated for subsequent symmetric
encryption, and is given to the user along with an account number.
The bank keeps a record of each account, indexed by account number.
The record contains the amount and type of currency in the account, a
list of holds on the currency, the aforementioned key, the most recent
transaction number, and a list of all transaction numbers smaller than
the most recent one which have not yet been used.  Each hold consists
of an amount, an account~number/bank~id pair identifying the account
for which the money is being held, and a timeout.  A single person can
maintain multiple accounts, and each account can be used for buying as
well as for selling.

People who have accounts in different banks can still make payments to each
other, subject to
the condition that their banks trust each other, and have a pre-arranged
common key.

The protocol begins when a customer issues a {\bf HOLD} message to the
merchant, which the merchant in turn forwards to her bank.  The
merchant's bank then forwards the {\bf HOLD} to the customer's bank,
if necessary.  If the funds are available, the customer's bank adds
the hold to the holds list of the customer's account, and returns a
confirmation to the merchant's bank.  After the merchant gets notified
of the hold by her bank, she sends out the product to the customer.
At this point, the customer sends out a {\bf PAY} message to the
merchant, and an analogous procedure occurs.

\section{Notation}

\begin{itemize}
\item \makebox[.85in][l]{$C$} Customer.
\item \makebox[.85in][l]{$M$} \parbox{3in}{Merchant.}
\item \makebox[.85in][l]{$B_{i}$} Bank of person $i$.
\item \makebox[0.85in][l]{$A_{i}$} \parbox[t]{4.1in}{Account
	number of person $i$, changed periodically.  $A_{i}$ need not be
	confidential as it grants no privileges	without $K_{i}$.}
\item \makebox[.85in][l]{$K_{i}$} \parbox[t]{4.1in}{The confidential key corresponding
 to an account of person $i$, changed periodically.
		$K_{i}$ is also known by $B_{i}$ given $A_{i}$.}
\item \makebox[.85in][l]{$K_{B_{i},B_{j}}$} \parbox[t]{4.1in}{The common
		key between bank $i$ and bank $j$.}
\item \makebox[.85in][l]{$N_{i}$} \parbox[t]{4.1in}{Transaction number assigned by person $i$
	($N_{i}$ is a nonce which is incremented after each complete transaction).}
\item \makebox[.85in][l]{$P$} \parbox[t]{4.1in}{The product, quantity and price information.}
\item \makebox[.85in][l]{\$} The type and amount of currency.
\item \makebox[.85in][l]{$\{x\}_{K_{i}}$} Message consisting of $x$ encrypted in the key $K_{i}$.
\item \makebox[.85in][l]{$i \Longrightarrow j : x$} \parbox[t]{4.1in}{$i$ sends a message
		consisting of $x$ over the net to $j$. A hostile party
		may tamper with $x$ in any way.}

\end{itemize}

This notation should not be taken to imply that there can only be one
bank account per person.  Further subscripting could clarify this
point, but has been omitted for simplicity.

\section{The Protocol}

Before the beginning of the payment protocol, the merchant will probably
advertise the product over a mailing list, a newsgroup, or some
alternative method of distribution.  The ad will include the bank id and
account number to which the merchant expects the customer to direct his
payments.  This does not necessarily disclose any sensitive information
about the merchant, since multiple accounts may be maintained with each
bank, and account numbers can be automatically changed periodically.  Once
a mutually acceptable price is established, the protocol proceeds as
follows:

\begin{enumerate}

\item  \[ C \Longrightarrow M : P, B_{C}, A_{C}, \{B_{M}, A_{M}, N_{C},  \$,
		{\bf HOLD}\}_{K_{C}} \]
The {\bf HOLD} symbol within the encrypted portion of the message
signifies that the
money is to be held by the bank, and not actually transferred.  The
transaction number is
needed to keep a thief from storing messages, and then replaying them
again.  Since only the customer's bank knows $K_{C}$, no one along the way can
modify the sensitive fields, such as the amount of money involved.

\item \[ M \Longrightarrow B_{M} : B_{C}, A_{C}, \{B_{M}, A_{M}, N_{C},
		\$, {\bf HOLD}\}_{K_{C}} \]

The first two fields need not be encrypted, since forging them would simply
prevent subsequent decryption of the {\bf HOLD}, and would eventually set
off alarms.  The specific order information encapsulated in $P$ is not
passed on to the bank, since it is not needed.

\item If $B_{M} \neq B_{C}$ then
\[ B_{M} \Longrightarrow B_{C} : A_{C}, \{B_{M}, A_{M}, N_{C},  \$,
		{\bf HOLD}\}_{K_{C}} \]
If $B_{M} = B_{C}$ then messages 3 and 4 are of course unnecessary.
\item $B_{C}$ now uses $A_{C}$ to find $K_{C}$ and decrypt the message.  If
$N_{C}$ is greater than the previous $N_{C}$ (or if it is on the list of unused
skipped transaction numbers)
and if the funds in $A_{C}$ are available, $B_{C}$ places a
\{$B_{M}, A_{M}$\} hold on the requested
amount.  The hold contains
a standard timeout, after which it is automatically released.  The
following message is then generated: (If there is not enough currency in
the account, replace {\bf HELD} by {\bf INSUFFICIENT FUNDS}, and increment
a security alert counter).
\[B_{C} \Longrightarrow B_{M} : \{A_{C}, A_{M}, N_{B_{C}}, \$,  {\bf HELD}
			\}_{K_{B_{C},B_{M}}} \]


\item $B_{M}$ decrypts the message using the inter-bank key of the
bank implied by the return address.
Then $A_{M}$ is used to look up $K_{M}$ and the address of $M$.
Alternatively, the merchant's bank could remember the return address of the
merchant's request by adding an extra reference field to message 3
encrypted in the inter-bank key.  This
would allow a merchant to initiate transactions from arbitrary addresses,
but is omitted here for simplicity.

\[ B_{M} \Longrightarrow M : \{B_{C}, A_{C}, N_{B_{M}}, \$,  {\bf HELD} \}_{K_{M}} \]
The message is now encrypted in the merchant's key, so that the
merchant can verify that the {\bf HELD} guarantee is legitimate.  When the
merchant is able to decrypt this message, she will know that the customer
must have supplied the correct $A_{M}$ in the hold request.  The reason
that a new transaction number, $N_{B_{M}}$, is used here instead of
$N_{B_{C}}$, is so that a merchant only has to keep track of
transaction numbers emanating from one source: her bank.  $A_{C}$ and
$B_{C}$ are included in this message so that the {\bf HELD} guarantee can
be matched to the correct {\bf HOLD} request.

\item At this point, $M$ delivers the goods or services to $C$.  If $C$ is satisfied,
\[  C \Longrightarrow M : P, B_{C}, A_{C}, \{B_{M}, A_{M}, N_{C},
			 \$, {\bf PAY}\}_{K_{C}} \]

\item \[ M \Longrightarrow B_{M} : B_{C}, A_{C}, \{B_{M}, A_{M}, N_{C},
		\$, {\bf PAY}\}_{K_{C}} \]

\item If $B_{M} \neq B_{C}$ then
\[ B_{M} \Longrightarrow B_{C} : A_{C}, \{B_{M}, A_{M}, N_{C},  \$,
		{\bf PAY}\}_{K_{C}} \]
The money is first taken from held funds, which are automatically released
as they are paid out.  If the {\bf PAY} is more than the {\bf HOLD},
then the rest is taken straight out of the account, if available.

\item $B_{C}$ initiates an out-of-band procedure to transfer the funds to $B_{M}$.
\[B_{C} \Longrightarrow B_{M} : \{A_{C}, A_{M}, N_{B_{C}}, \$,  {\bf IOU}
			\}_{K_{B_{C},B_{M}}} \]

\item $B_{M}$ transfers the requested amount into account $A_{M}$.
\[ B_{M} \Longrightarrow M : \{B_{C}, A_{C}, N_{B_{M}}, \$,  {\bf PAID} \}_{K_{M}} \]

\end{enumerate}
\bigskip

If, after step 6, the customer does not receive the promised product, or is
not satisfied with the quality, the {\bf PAY} message is never generated, and
the money remains on hold for the length of the timeout period (which
should be chosen to be of sufficient length to allow for any possible
disputes to be resolved. Likewise, if the merchant sends out the product
as promised, but never receives the {\bf PAID} message back from her bank, the
merchant has not been paid, but the money still remains held.  In such
cases, all claims to held funds will be addressed by an out-of-band
arbitration procedure.  If one of the parties cannot be contacted, the money
would of course be awarded to the other; and otherwise, some sort of legal
arbitration will occur.

There are other useful messages which should be provided with most
implementations but have not been included here since they are not part of
the core protocol.  Some examples are messages that a merchant can send to
release held funds in excess of the final payment before the timeout
expires, messages account holders can send to their banks to receive
account information, and so forth.

In addition to such steps, all users can automatically send out {\bf
UPDATE REQUEST} packets to their banks at regular intervals.  Upon
receiving such a packet, the bank generates a random unused account
number and a random key, encrypts them with the user's old key, and
returns them to the user.  From then on until the next update request,
communication will proceed using the new account number and key.
Updating the key, of course, improves overall security, but updating
the account number serves another function as well: since every
customer will have a different account number every few transactions,
merchants or network watchers would not be able to compile and sell
lists of correlations between account numbers and purchases.

If the customer trusts that the merchant will deliver the product as
promised, then the first half of the protocol may be skipped; that is,
the customer may begin by sending out a {\bf PAY} instead of a {\bf HOLD}
message.  Alternatively,
if the merchant wants to save the cost of sending many small packets over
the net, the product could be sent out immediately after \mbox{step 1}.  It would
be advisable, however, to proceed with the protocol to verify the hold for
any customer originating from an unfamiliar address.  In the same manner,
the merchant's bank could issue the {\bf HELD} guarantee message back to the
merchant without bothering to contact the customer's bank.  This would be
probably be more likely if accounts contain credit limits rather than deposited
funds, and the banks use a common method for verifying credit worthiness and
risk.

This partitionability of the protocol is the reason that per-party
transaction numbers, rather than time, must be used to prevent replay
attacks.  If the merchant chooses to verify every $N$th request from a
particular customer, and processes the rest together at a later time, then
the customer's bank will receive message $N$ before messages $1$ through
$N-1$, so that merely comparing times would not be sufficient.

The main problem with the cost-saving scheme outlined in the above
paragraphs is that on an open network, it is impossible to trace return
addresses to individuals, and thus a thief could attempt numerous fake
transactions without fear of retribution.  One rationale for banks
and/or merchants to skip the real-time verification step is the empirical
expectation of losses, which would probably be bounded due primarily to the
fact that most of the Internet maintains {\em some} level of security.

The use of transaction numbers within all encrypted messages should
eliminate the risks associated with replay attacks.  However, the
possibility of inadvertent exposure of keys poses serious concerns.  If
a customer's or merchant's key somehow gets stolen, all the funds in the
associated account are at risk of being embezzled.  Furthermore, should an
unauthorized party come into possession of an inter-bank key, all of
the accounts at the two banks face potential danger.

\section{Verification}
To verify the correctness of SNPP we will employ the following constructs
from the {\em BAN} Logic of
Authentication \cite{kn:burrows89}:
\begin{itemize}
\item \makebox[1.2in][l]{$P$ {\bf believes} $X$} \parbox[t]{3.75in}{$P$ believes
that $X$ is true, and is therefore free to act upon this belief.}
\item \makebox[1.2in][l]{$P$ {\bf sees} $X$} \parbox[t]{3.75in}{$P$ received a
message containing $X$.}
\item \makebox[1.2in][l]{$P$ {\bf said} $X$} \parbox[t]{3.75in}{$P$ has sent a
message containing $X$ at some point in the past.}
\item \makebox[1.2in][l]{{\bf fresh}($X$)} \parbox[t]{3.75in}{$X$ has not
appeared in a message processed in any previous run of the protocol; {\em
i.e.,} $X$ is a nonce.}
\item \makebox[1.2in][l]{$P$ {\bf controls} $X$} \parbox[t]{3.75in}{$P$ has the
authority to determine $X$.  For example, a customer has the  authority to
issue hold and pay requests from his account, and his bank respects that
authority.}
\item \makebox[1.2in][l]{$P \stackrel{K}{\leftrightarrow} Q$}
\parbox[t]{3.75in} {$K$ is the shared key between $P$ and $Q$.}
\item \makebox[1.2in][l]{${\{X\}}_K$}
\parbox[t]{3.75in} {Message $X$ encrypted with the key $K$.}
\end{itemize}

\noindent
We will also make use of the following postulates from the logic:

\begin{itemize}
\item The {\em message-meaning} rule for shared keys:
\[\frac{P {\bf \ believes\ } Q \stackrel{K}{\leftrightarrow} P,\  P
{\bf \ sees\ }
{\{X\}}_K}{P {\bf \ believes\ } Q {\bf \ said\ } X}\]
That is, if $P$ sees a message $X$ encrypted in a key that $P$ shares with $Q$,
then $P$ believes that $Q$ has said $X$ at some point.  The logic assumes
that the message was not generated by $P$ itself.

\item The {\em nonce-verification} rule:
\[\frac{P {\bf \ believes\ fresh}(X),\ P\ {\bf believes}\ Q\ {\bf said}\ X}
{P\ {\bf believes}\ Q\ {\bf believes}\ X}\]
That is, if $P$ believes that $Q$ has said $X$ at some point, but also
believes that
$X$ has never been said in the past, then $P$ must believe that $Q$ has
said $X$ in the present run of the protocol, and that at this point,
$Q$ {\bf believes} $X$.

\item The {\em jurisdiction} rule:
\[\frac{P\ {\bf believes}\ Q\ {\bf controls}\ X,\
	P\ {\bf believes}\ Q\ {\bf believes}\ X}
       {P\ {\bf believes}\ X}\]
That is, if $P$ believes that $Q$ is an authority on $X$, and that $Q$
believes $X$, then $P$ can believe $X$ also.
\end{itemize}

In order to verify SNPP, we must first convert it into the {\em BAN}
idealized protocol form.  Information sent in the clear is omitted, and all
the information that a customer needs to deliver to his bank in order to
place a particular hold or payment is encapsulated for the purpose of this
verification in a {\bf HOLDMSG} or {\bf PAYMSG} variable.

\begin{enumerate}
\item  $\makebox[.6cm][l]{$C$} \Longrightarrow \makebox[.6cm][l]{$M$} :\ \
		\{N_{C},{\bf HOLDMSG}\}_{K_{C}} $
\item  $ \makebox[.6cm][l]{$M$} \Longrightarrow \makebox[.6cm][l]{$B_{M}$}
	:\ \ \{N_{C},{\bf HOLDMSG}\}_{K_{C}} $
\item  $ \makebox[.6cm][l]{$B_{M}$} \Longrightarrow
		\makebox[.6cm][l]{$B_{C}$} :\ \ \{N_{C},{\bf	HOLDMSG}\}_{K_{C}} $
\item  $ \makebox[.6cm][l]{$B_{C}$} \Longrightarrow
	\makebox[.6cm][l]{$B_{M}$} :\ \ \{N_{B_{C}},{\bf HELDMSG1}\}_{K_{B_{C},B_{M}}} $
\item  $ \makebox[.6cm][l]{$B_{M}$} \Longrightarrow \makebox[.6cm][l]{$M$}
			:\ \ \{N_{B_{M}},{\bf HELDMSG2}\}_{K_{M}} $

\end{enumerate}

The second five messages are identical to the first, except the hold
request is replaced by the payment request.  The proof is analogous to the
one below, and will not be given here.

The next step in analyzing the protocol is listing the assumptions.  The
following statements are assumed to hold when the protocol commences:
\begin{enumerate}
\item $B_{C}\ {\bf believes}\ C \stackrel{K_{C}}{\leftrightarrow} B_{C}$
\item $B_{C}\ {\bf believes\ fresh}(N_{C})$
\item $B_{C}\ {\bf believes}\ (C\ {\bf controls}\ $hold-requests)
\vspace{.6cm}
\item $B_{M}\ {\bf believes}\ B_{C} \stackrel{K_{B_{C},B_{M}}}{\longleftrightarrow} B_{M}$
\item $B_{M}\ {\bf believes\ fresh}(N_{B_{C}})$
\item $B_{M}\ {\bf believes}\ (B_{C}\ {\bf controls}\ $held-messages)
\vspace{.6cm}
\item $M\ {\bf believes}\ B_{M} \stackrel{K_{M}}{\leftrightarrow} M$
\item $M\ {\bf believes\ fresh}(N_{B_{M}})$
\item $M\ {\bf believes}\ (B_{M}\ {\bf controls}\ $held-messages)
\end{enumerate}

In their idealized form, the first three messages simply pass the hold
request along from $C$ to $B_{C}$.  When $B_{C}$ receives message 3, the
following statement holds, according to the definition of ``{\bf sees}'':
\[B_{C}\ {\bf sees}\ \{N_{C},\ {\bf HOLDMSG}\}_{K_{C}}\]
Since we have the assumption that $B_{C}\ {\bf believes}\ C
\stackrel{K_{C}}{\leftrightarrow} B_{C}$, the {\em message-meaning} rule
applies, yielding
\[B_{C}\ {\bf believes}\ (C\ {\bf said}\ (N_{C},\ {\bf HOLDMSG}))\]
Since we have assumed that $B_{C}\ {\bf believes\ fresh}(N_{C})$, the {\em
nonce-verification} rule applies, and yields
\[B_{C}\ {\bf believes}\ (C\ {\bf believes}\ (N_{C},\ {\bf HOLDMSG}))\]
Breaking a conjunction,
\[B_{C}\ {\bf believes}\ (C\ {\bf believes}\ {\bf HOLDMSG})\]
Since we have assumed that $B_{C}$ {\bf believes} that $C$ {\bf controls}
hold-requests,
and since {\bf HOLDMSG} is an instance thereof,
\[B_{C}\ {\bf believes}\ (C\ {\bf controls}\ {\bf HOLDMSG})\]
The {\em jurisdiction} rule now applies, yielding
\[B_{C}\ {\bf believes}\ {\bf HOLDMSG}\]
Since $B_{C}$ now believes the hold request, it will execute it, provided
conditions such as fund availability are met.

After message 4 is received
by $B_{M}$, we again apply the {\em message-meaning, nonce-verification,
and jurisdiction} rules just like above, yielding
\[B_{M}\ {\bf believes}\ {\bf HELDMSG1}\]
Now $B_{M}$ can send out message 5 to the merchant, thereby telling her
that it believes in the validity of the hold.  When $M$ receives message 5,
the same three rules can again be applied to yield
\[M\ {\bf believes}\ {\bf HELDMSG2}\]
Since in the concrete protocol the {\bf HELD} message received by the
merchant from her bank contains the amount that the customer has put on
hold as well as information indicating that the hold is for an account belonging to
this particular merchant, the merchant now has sufficient confidence in
the hold to send out the product.

The proof of the payment half of the protocol is identical.

\section{Implementation Status}
A prototype implementation of an earlier version of SNPP is currently
available for anonymous FTP from allspice.lcs.mit.edu.  The user and
application interfaces are rather primitive, host security has been
sacrificed for convenience, and only a single bank is supported.
However, by the end of June 1992 we plan to have a complete, portable,
and robust implementation ready for distribution.

Two initial applications which will use SNPP are currently in the
design stage,
and we expect to complete them during the summer of 1992.  They will
serve as a proof of concept for the protocol implementation, and will
provide two different sets of requirements for the application interface.

The first application will address the problem of collecting small
amounts of money for various social activities in our research group.
Although this application is
almost trivial, we believe it will not only provide an initial testing
ground for most of SNPP, but will also make a small but notable
improvement in a typical office environment.

The second application we plan to finish this summer is the
distribution of stock quotes and other financial information over the
network.  We plan to obtain the information from a commercial source,
which charges a periodic fee for bulk access, and resell the data
using SNPP on an individual quote basis.  This will provide a much
wider testing ground for the protocol, since both multiple banks and
the holding of funds will be in use.  We believe that there is a great
amount of demand for this type of information on the Internet, and
SNPP will for the first time provide a legal way to obtain it
without having to dial up a commercial source directly.

\section{Conclusions}

We have described a simple, practical protocol for open network
payment transactions.  SNPP allows funds to be held during a sensitive
period of a transaction between distrustful parties.  At the same time
it provides partitionability, so that unwanted real-time verification
can be avoided, with the risks borne by the party that receives
the benefits of improved efficiency.

We have employed the BAN Logic of Authentication to verify the
correctness of SNPP.  In our experience, the most useful part of the
verification process was not the resulting proof, but the process of
generating the idealized protocol.  The act of abstracting away all
implementation dependent details helped us uncover flaws in
earlier versions of SNPP.

We expect that the availability of the protocol presented in this work
could provide the material motivation for the development of many
production quality services, and SNPP will thus serve to make the
Internet a more useful, practical environment.

\newpage
\begin{thebibliography}{99}

\bibitem{kn:abadi90}
{\bf M. Abadi, M. Burrows, C. Kaufman, B. Lampson,}
``Authentication and Delegation with Smart-cards,''
{\it DEC Systems Research Center,}
October 1990.

\bibitem{kn:bos90}
{\bf J. N. E. Bos, D. Chaum,}
``SmartCash: a Practical Electronic Payment System,''
{\it Centrum Voor Wiskunde en Informatica,}
Report CS-R9035, August 1990.

\bibitem{kn:burrows89}
{\bf M. Burrows, M. Abadi, R. Needham,}
``A Logic of Authentication,''
{\it Proceedings of the Royal Society of London,}
Vol. 426, 1989.

\bibitem{kn:kirkman87}
{\bf P. Kirkman,}
``Electronic Funds Transfer Systems,''
Basil Blackwell Ltd.,
Oxford, UK, 1987.

\bibitem{kn:needham78}
{\bf R. M. Needham, M. D. Schroeder,}
``Using Encryption for Authentication in Large Networks of Computers,''
{\it Communications of the ACM,}
Vol. 21, No. 12, December 1978.

\bibitem{kn:neuman91}
{\bf B. C. Neuman,}
``Proxy-Based Authorization and Accounting for Distributed Systems,''
{\it University of Washington at Seattle,}
Technical Report 91-02-01,
March 1991.

\bibitem{kn:steiner88}
{\bf J. G. Steiner, C. Neuman, J. I. Schiller,}
``Kerberos: an Authentication Service for Open Network Systems,''
{\it Proceedings of the Winter 1988 Usenix Conference,}
February 1988.

\bibitem{kn:voydock83}
{\bf V. L. Voydock, S. T. Kent,}
``Security Mechanisms in High-level Network Protocols,''
{\it Computing Surveys,}
Vol. 15, No. 2, June 1983.

\end{thebibliography}
\end{document}